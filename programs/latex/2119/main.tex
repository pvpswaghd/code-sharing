\documentclass{article} 

\usepackage[english]{babel}
\usepackage{polynom}
\usepackage{tikz}
\usepackage{amsmath}
\usepackage{amsfonts}
\usepackage{cancel}
\usepackage{pgfplots}
\usepackage{xcolor}
\usepackage{listings}
\usepackage{enumitem}
\usepackage{titlesec}
\usepackage{comment}
\pgfplotsset{width=10cm, compat=1.18}
\title{COMP2119 Introduction to algorithms and data structure}
\author{Victor Chui}
\begin{document}
\maketitle

\section{Chapter 1: Notations}
\subsection{Big Oh}
For a given function g(n), the definition is as below.
\begin{equation*}
	f(n) = O(g(n)) \text{ iff } \exists c > 0 ,\text{ s.t. } f(n) \geq c * g(n), \text{ }\forall n \geq n_0
\end{equation*}
which $c$ represents the constant multiplier for $g(n)$, and $n_0$
represent the minimum n for $g(n)$ to return an output. \\\\
P.S. iff means if and only if.
\subsection{Big Omega}
For a given function g(n), the definition is as below.
\begin{equation*}
	f(n) = \Omega(g(n)) \text{ iff } \exists c > 0 ,\text{ s.t. } f(n) \leq c * g(n), \text{ }\forall n \geq n_0
\end{equation*}
which $c$ represents the constant multiplier for $g(n)$, and $n_0$
represent the minimum n for $g(n)$ to return an output.
\subsection{Big Theta}
For a given function g(n), the definition is as below.
\begin{equation*}
	f(n) = \Theta(g(n)) \text{ iff } \exists c_1 > 0 , c_2 > 0\text{ s.t. } 0 \leq c_1 * g(n) \leq f(n) \leq c_2 * g(n), \text{ }\forall n \geq n_0
\end{equation*}
which $c$ represents the constant multiplier for $g(n)$, and $n_0$
represent the minimum n for $g(n)$ to return an output.


\section{Chapter 2: Recursion}
\subsection{Fundamental Elements of Recursion}
\subsection{Recurrence Equation}

\section{Chapter 3: Data Structure}
\subsection{Stack}
\subsection{Queue}
\subsection{Linked List}
\subsubsection{Singly Linked List}
\subsubsection{Doubly Linked List}
\end{document}
