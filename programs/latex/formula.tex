\documentclass{article} 

\usepackage[english]{babel}
\usepackage{amsmath}
\usepackage{amsfonts}
\usepackage{cancel}
\usepackage{pgfplots}
\usepackage{xcolor}
\usepackage{listings}
\usepackage{hyperref}
\usepackage{enumitem}
\usepackage{titlesec}
\usepackage{comment}
\pgfplotsset{width=10cm, compat=1.18}
\title{1310 Formula list 1.0}
\author{Victor Chui}
\begin{document}
\maketitle
\tableofcontents
\pagebreak
\section{Semiconductor}
\subsection{Basic definitions}
To find the density of electrons in a material, we have
\begin{equation*}
	n_i = 5.2*10^{15}T^{\frac{3}{2}}exp\frac{-E_g}{2kT}\text{ electrons/}cm^3
\end{equation*}
In both intrinsic and extrinsic conductors, the electron density and hole density is equal. Thus:
\begin{equation*}
	np = n_i^2
\end{equation*}
where $n_i$ is the densities of intrinsic material.\\\\
For a p-type semiconductor, holes are the majority carrier and electrons are the minority carrier. Thus,
\begin{equation*}
	\text{Majority Carriers: } n \approx N_D \text{ which }N_D\text{ is the density of donor atoms}
\end{equation*}
\begin{equation*}
	\text{Minority Carriers: } p \approx \frac{n_i^2}{N_D}
\end{equation*}
Similarly, for a density of $N_A$ acceptor atoms per cubic centimeter, we have:
\begin{equation*}
	\text{Majority Carriers: } p \approx N_A 
\end{equation*}
\begin{equation*}
	\text{Minority Carriers: } n \approx \frac{n_i^2}{N_A}
\end{equation*}
\subsection{Drift}

\end{document}
