\documentclass{exam} 

\usepackage[english]{babel}
\usepackage{polynom}
\usepackage{tikz}
\usepackage{amsmath}
\usepackage{amsfonts}
\usepackage{cancel}
\usepackage{pgfplots}
\usepackage{xcolor}
\usepackage{listings}
\usepackage{hyperref}
\usepackage{enumitem}
\usepackage{titlesec}
\usepackage{comment}
\usepackage{multicol}
\usepackage{blindtext}
\pgfplotsset{width=10cm, compat=1.18}
\begin{document}
\begin{multicols}{2}
[
\section{Differentiation}
]
\noindent
$\frac{d}{dx}\sin x = \cos x$ \\\\
$\frac{d}{dx}\cos x = -\sin x$ \\\\
$\frac{d}{dx}\tan x = \sec ^2x$ \\\\
$\frac{d}{dx}\sec x = \sec x\tan x$ \\\\
$\frac{d}{dx}\csc x = -\csc x\cot x$ \\\\
$\frac{d}{dx}\cot x = -\csc ^2x$ \\
\end{multicols}
\begin{multicols}{2}
[
\section{Formulas}
]
$\sin 2x = $\\\\
$\cos 2x = $\\\\
$\sin ^2x + \cos^2x = 1$\\\\
Pythagorean Identity (tan): $\tan^2 x + 1 = \sec^2 x$\\\\
Pythagorean Identity (cot): $\sec^2 x - 1 = \tan^2 x$\\\\
\end{multicols}
\begin{multicols}{2}
[\section {Theorem}]
Fundamental Theorem of Calculus (Part 1):\\ If f is continuous at [a,b], then $\int_a^x f(t)dt$ is continuous on [a,b] and differentiable at (a,b), and the derivative is \\$F'(x) = \frac{d}{dx} \int_a^x f(t)dt = f(x)$ \\\\
Mean Value Theorem:\\
If f is continous at [a,b],
differentiable at (a,b), 
then there must be a $c \in (a,b)$ which\\
$f'(c) = \frac{f(b)-f(a)}{b-a}$\\\\
Intermediate Value Theorem: \\
If f is proved continuous as [a,b], then let M be a number which lies between f(a) and f(b), there will be a number c $\in$ [a,b] such that $f(c) = M$. \\\\\\
\end{multicols}
\begin{multicols}{2}
[
\section{Laplace Transform}
]
$\frac{1}{(s-a)(s-b)} = \frac{A}{s-a} + \frac{B}{s-b}$ \\\\\\
$\frac{1}{(s-a)^3} = \frac{A}{s-a} + \frac{B}{(s-a)^2} + \frac{C}{(s-a)^3}$ \\\\\\
$\frac{1}{(s-a)^2+b^2} = \frac{A(s-a)}{(s-a)^2+b^2} + \frac{B(b)}{(s-a)^2+b^2}$ \\\\\\
Convolution: $f * g = \int_0^t f(u)g(t-u)du$\\\\\\
First shift: \\\\\\
Second shift: \\\\
\end{multicols}
\newpage
\begin{multicols}{2}
[
\section{Areas}
]
Polar Equation: $r = \sin\theta$\\\\\\
Area of Polar Equation: $A = \int_a^b \frac{1}{2}r^2d\theta$\\\\
Volume of Solid: \\\\\\
\end{multicols}
\section{ODE}
\begin{questions}
\question Solve the following ODE using D-operator. 
\begin{equation*}
	y''+2y'+y=e^{-x}
\end{equation*}
Solution:\\
Let $y = e^{rx},$\\
\begin{align*}
	r^2+2r+1&=0\\
	(r+1)^2&=0
\end{align*}
Therefore,
\begin{equation*}
	y = C_1e^{-x}+C_2xe^{-x}
\end{equation*}
Write $D = \frac{d}{dx}$:
\begin{align*}
	D^2y+2Dy+y &=e^{-x} \\
	(D^2+2D+1)y &= e^{-x} 
\end{align*}
Let $f = e^{-x}$.
\begin{align*}
	Df &= -e^{-x} \\
	D^2f &= e^{-x} \\
	D^2f - f &= 0 \\
	(D^2 - 1)f &= 0
\end{align*}
Back to the equation,
\begin{align*}
	(D^2+2D+1)(D^2-1)y &= 0 \\
	(D+1)^2(D+1)(D-1)y &= 0
\end{align*}
\question Solve the following Cauchy-Euler equation:
\begin{equation*}
	x^2y''+3xy'+6y=0
\end{equation*}
\question Find the Laplace Transform of
\begin{equation*}
	t\sin bt
\end{equation*}
\end{questions}
\end{document}

