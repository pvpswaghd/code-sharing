\documentclass{article}

\title{1310 Assignment 3 Submission}
\author{Chui Wai Chun Victor, 3035927398 BEng(CompSc)}
\begin{document}
\maketitle
\noindent
1a) \\
0 0 0 -> 0\\
0 0 1 -> 0\\
0 1 0 -> 0\\
0 1 1 -> 1\\
1 0 0 -> 1\\
1 0 1 -> 1\\
1 1 0 -> 0\\
1 1 1 -> 1\\\\
1b)\\
In respect to figure 1 and table 1, there's no input that are different with table. However, without considering the above figure and table, the result of 1 OR 1 versus 1 XOR 1 is different, where the former outputs 1 and the latter outputs 0.\\
\\
2)\\
Usage: Main memory such as RAM is used to hold data or applications which can be directly accessed by the processing unit with minimum or no delay. On the contrary, secondary storage is used to store and retrieve data permanently with no delay.\\\\
Status when off: Main memory is a volatile memory which means that data is lost when it loses power, while secondary storage is a non-volatile memory which is able to retain data even if it loses power.\\\\
Storage capacity and speed: Main memory usually has a relatively low storage capacity but high speed compared to secondary storage which has a relatively high storage capacity with lower access speed.
\\
\\
3)\\
1. Arithmetic and Logical unit (ALU): It performs predefined functions according to the instruction, such as bit operation, arithmetic operation, comparison and so on.\\\\
2. Control unit: It fetches instructions and parse it to functions needed to be finished for ALU. Besides, it performs memory read/write operation as well as a reset of the datapath if requested.\\\\
3. Register: It is a small but high-speed memory in CPU. It stores a small amount of data, for instance the address of the instruction, the current instruction, the result of various calculations that are needed during processing.\\\\
\end{document}
