\documentclass{article} 
\usepackage[english]{babel}
\usepackage{amsmath}
\usepackage{amsfonts}
\usepackage{cancel}
\usepackage{pgfplots}
\usepackage{xcolor}
\usepackage{listings}
\usepackage{hyperref}
\usepackage{enumitem}
\usepackage{titlesec}
\usepackage{comment}

\title{Formulas for 1851 exams}
\author{Victor Chui}
\begin{document}
\maketitle
\begin{equation}
	\sin(A\pm B) = \sin A\cos B\pm \cos A\sin B	
\end{equation}
\begin{equation}
	\cos(A\pm B) = \cos A\cos B\mp \sin A\sin B
\end{equation}
\begin{equation}
	\tan(A\pm B) = \frac{\tan A\pm \tan B}{1\mp \tan A\tan B}
\end{equation}
\begin{equation}
	\sin 2A = 2\sin A\cos A
\end{equation}
\begin{equation}
	\cos 2A = \cos^{2}A - \sin^{2}A = 1-2\sin^{2}A
\end{equation}
\begin{equation}
	\tan 2A = \frac{2\tan A}{1-\tan^{2}A}
\end{equation}
\begin{equation}
	2\sin A\cos B = \sin(A+B)+\sin(A-B)
\end{equation}
\begin{equation}
	2\cos A\cos B = \cos (A+B)+\cos (A-B)
\end{equation}
\begin{equation}
	2\sin A\sin B = 2\sin \frac{A+B}{2} \cos \frac{A-B}{2}
\end{equation}
\begin{equation}
	\sin A + \sin B = 2\sin \frac{A+B}{2} \cos \frac{A-B}{2}
\end{equation}
\begin{equation}
	\sin A - \sin B = 2\cos \frac{A+B}{2} \sin \frac{A-B}{2}
\end{equation}
\begin{equation}
	\cos A + \cos B = 2\cos \frac{A+B}{2} \cos \frac{A-B}{2}
\end{equation}
\begin{equation}
	\cos A - \cos B = -2\sin \frac{A+B}{2} \sin \frac{A-B}{2}
\end{equation}
\begin{equation}
	\int \sec xdx = \ln(\sec x + \tan x) + C
\end{equation}
\end{document}
